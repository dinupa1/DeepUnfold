% dinupa3@gmail.com
% 26-Jan-2023
%

\documentclass[10pt, xcolor={dvipsnames}, aspectratio = 169, sans,mathserif]{beamer}

\usepackage{fontspec}
\usepackage{fontawesome5}
\usepackage{mathrsfs}
\usepackage{amsmath}
\usepackage{graphicx}
\usepackage{hyperref}
\usepackage[absolute,overlay]{textpos}
\usepackage[font=tiny]{caption}


\mode<presentation>
{
\usefonttheme{serif}
\setmainfont{JetBrains Mono}
\definecolor{nmsured}{RGB}{137,18,22} % custom colors
\setbeamercolor{title}{bg=White,fg=nmsured}
\setbeamercolor{frametitle}{bg=White,fg=nmsured}
\setbeamercolor{section number projected}{bg=nmsured,fg=White}
\setbeamercolor{subsection number projected}{bg=nmsured,fg=White}
\setbeamertemplate{items}{\color{nmsured}{\faAngleDoubleRight}}
\setbeamertemplate{section in toc}[square]
\setbeamertemplate{subsection in toc}[square]
\setbeamertemplate{footline}[frame number]
\setbeamertemplate{caption}[numbered]
\setbeamerfont{footnote}{size=\tiny}
\setbeamercovered{invisible}
\usefonttheme{professionalfonts}
%\setbeamertemplate{background}[grid][color=nmsured!15] % set background
\setbeamertemplate{navigation symbols}{} % remove navigation buttons
}

\title{Just an Idea}


\newcommand{\leftpic}[2]
{
\begin{textblock}{7.0}(0.5, 1.0)
\begin{figure}
    \centering
    \includegraphics[width=7.0cm]{../imgs/#1.png}
    \caption{#2}
\end{figure}
\end{textblock}
}

\newcommand{\rightpic}[2]
{
\begin{textblock}{7.0}(8.0, 1.0)
\begin{figure}
    \centering
    \includegraphics[width=7.0cm]{../imgs/#1.png}
    \caption{#2}
\end{figure}
\end{textblock}
}


\begin{document}

\begin{frame}
    \maketitle
\end{frame}

\begin{frame}[fragile]{Input Histograms}

\leftpic{train_hist}{2D histogram used for training.}

\rightpic{gray_toy}{Image of the 2D histogram. This is a conisdered as a 20 x 20 pixel imgage. Each histogram is scaled by the \href{https://scikit-learn.org/stable/modules/generated/sklearn.preprocessing.StandardScaler.html}{standard scalar}.}

\begin{textblock}{15.0}(0.5, 12.0)
\begin{itemize}

    \item 100K histograms were generated randomly with $\lambda, \mu, \nu$ (as targets) in range [-1.0, 1.0] and they are split in to test: validate: train = 60: 20: 20.

\end{itemize}
\end{textblock}
\end{frame}

\begin{frame}{CNN Architecture}
\begin{textblock}{15.0}(0.5, 2.0)
\begin{itemize}

    \item Feature extraction;

    \begin{itemize}
        \item 2 convolutional layers.
        \item 2 max pooling layers.
        \item activated by ReLu activation function.
    \end{itemize}

    \item Regression layers;

    \begin{itemize}
        \item 3 linear layers.
        \item Activated by ReLu activation function.
    \end{itemize}

    \item Learning rate = 0.001 and L2 regulation = 1.0e-05.

    \item DNN was trained for 50 epochs.
\end{itemize}
\end{textblock}
\end{frame}

\begin{frame}{Loss Curve}

\leftpic{loss_curve}{Loss curve.}

\end{frame}

\begin{frame}{Testing}

\begin{textblock}{5.0}(0.5, 2.0)
\begin{figure}
    \centering
    \includegraphics[width=5.0cm]{../imgs/test_lambda.png}
    \caption{Test $\lambda$ distribution.}
\end{figure}
\end{textblock}

\begin{textblock}{5.0}(5.0, 2.0)
\begin{figure}
    \centering
    \includegraphics[width=5.0cm]{../imgs/test_mu.png}
    \caption{Test $\mu$ distribution.}
\end{figure}
\end{textblock}

\begin{textblock}{5.0}(10.0, 2.0)
\begin{figure}
    \centering
    \includegraphics[width=5.0cm]{../imgs/test_nu.png}
    \caption{Test $\nu$ distribution.}
\end{figure}
\end{textblock}

\begin{textblock}{15.0}(0.5, 11.0)

\begin{itemize}
    \item Results are promising ?
\end{itemize}

\end{textblock}

\end{frame}

\begin{frame}{$[\lambda, \mu, \nu] = [0.5, 0.0, 0.0] \sigma = 0.1$}

\begin{textblock}{5.0}(0.5, 2.0)
\begin{figure}
    \centering
    \includegraphics[width=5.0cm]{../imgs/test_lambda3.png}
    \caption{Test $\lambda$ distribution.}
\end{figure}
\end{textblock}

\begin{textblock}{5.0}(5.0, 2.0)
\begin{figure}
    \centering
    \includegraphics[width=5.0cm]{../imgs/test_mu3.png}
    \caption{Test $\mu$ distribution.}
\end{figure}
\end{textblock}

\begin{textblock}{5.0}(10.0, 2.0)
\begin{figure}
    \centering
    \includegraphics[width=5.0cm]{../imgs/test_nu3.png}
    \caption{Test $\nu$ distribution.}
\end{figure}
\end{textblock}

\end{frame}

\begin{frame}{$[\lambda, \mu, \nu] = [0.5, 0.0, 0.0] \sigma = 0.05$}

\begin{textblock}{5.0}(0.5, 2.0)
\begin{figure}
    \centering
    \includegraphics[width=5.0cm]{../imgs/test_lambda2.png}
    \caption{Test $\lambda$ distribution.}
\end{figure}
\end{textblock}

\begin{textblock}{5.0}(5.0, 2.0)
\begin{figure}
    \centering
    \includegraphics[width=5.0cm]{../imgs/test_mu2.png}
    \caption{Test $\mu$ distribution.}
\end{figure}
\end{textblock}

\begin{textblock}{5.0}(10.0, 2.0)
\begin{figure}
    \centering
    \includegraphics[width=5.0cm]{../imgs/test_nu2.png}
    \caption{Test $\nu$ distribution.}
\end{figure}
\end{textblock}

\end{frame}

\begin{frame}{$[\lambda, \mu, \nu] = [1.0, 0.0, 0.0] \sigma = 0.01$}

\begin{textblock}{5.0}(0.5, 2.0)
\begin{figure}
    \centering
    \includegraphics[width=5.0cm]{../imgs/test_lambda1.png}
    \caption{Test $\lambda$ distribution.}
\end{figure}
\end{textblock}

\begin{textblock}{5.0}(5.0, 2.0)
\begin{figure}
    \centering
    \includegraphics[width=5.0cm]{../imgs/test_mu1.png}
    \caption{Test $\mu$ distribution.}
\end{figure}
\end{textblock}

\begin{textblock}{5.0}(10.0, 2.0)
\begin{figure}
    \centering
    \includegraphics[width=5.0cm]{../imgs/test_nu1.png}
    \caption{Test $\nu$ distribution.}
\end{figure}
\end{textblock}


\end{frame}

\begin{frame}{Other}

\begin{textblock}{15.0}(0.5, 2.0)

\begin{itemize}

    \item

\end{itemize}

\end{textblock}

\end{frame}

\end{document}
